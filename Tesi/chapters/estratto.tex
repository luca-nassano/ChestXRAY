\chapter*{Estratto in lingua Italiana}
\label{cha:estratto}
\markboth{Estratto}{}
\addcontentsline{toc}{chapter}{Estratto in lingua Italiana}

Le radiografie del torace, data la loro semplicità e natura non invasiva, sono uno degli strumenti più comunemente utilizzati per la rilevazione di molte malattie. Negli ultimi anni la comunità scientifica ha pubblicato diversi dataset contenenti radiografie toraciche, accompagnate da informazioni aggiuntive riguardo la presenza o l’assenza di diverse patologie.
Questo lavoro si pone il problema di analizzarle, con l'obiettivo di costruire un sistema in grado di verificare automaticamente se un paziente soffre o meno di una determinata patologia. Inoltre, cerca di risolvere il cosiddetto problema della scatola nera, relativo a molte applicazioni di Machine Learning, legato al fatto che spesso queste ultime sono in grado di prendere una decisione ma non sono in grado di fornire una spiegazione riguardante i motivi che hanno guidato quella scelta.
Questo lavoro cerca di risolvere questo problema, producendo, in aggiunta alla previsione, una mappa di calore che evidenzi la regione affetta dalla malattia, insieme ad un riquadro che la circondi. In particolare, in questo studio indaghiamo diversi approcci, partendo dalle reti neurali di convoluzione, ampiamente impiegate in altre opere correlate a questa, sino alle meno utilizzate Random Forests, addestrate utilizzando i cosiddetti embedding, una rappresentazione dell'input di dimensioni ridotte.
Proponiamo inoltre una nuova tecnica per combinare le diverse previsioni, che sfrutta l'incertezza del singolo modello per assegnargli un peso adeguato durante la fase di aggregazione. Sebbene non siamo stati in grado di superare le prestazioni di alcuni lavori simili a questo, abbiamo ottenuto un AUROC medio di 0.902 calcolato su cinque diverse patologie, dimostrando che si possono ottenere buoni risultati anche con modelli che sono computazionalmente più efficienti e che richiedono molto meno tempo per essere addestrati rispetto alle reti neurali di convoluzione.
