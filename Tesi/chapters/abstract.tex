\chapter*{Abstract}
\label{cha:abstract}
\markboth{Abstract}{}
\addcontentsline{toc}{chapter}{Abstract}

Chest X-Rays, due to their simplicity and non-invasive nature, are one of the most common tools used for the detection of many diseases. During the last years the scientific community published different datasets containing CXR images, along with side information related to the presence or absence of different pathologies.
This work studies the problem of analyzing them, with the aim of building an automatic system able to check whether a patient suffer from a given disease or not. Moreover, it tries to overcome the so called black-box problem, an issue connected to many machine learning applications, related to the fact that they are able to provide a decision but they’re not capable of giving an explanation behind such decision. This work tries to solve it by producing, along with the prediction, a heatmap highlighting the region that most probably is affected by the disease and a bounding box surrounding it. In particular, in this work we investigate different approaches, ranging from Convolutional Neural Network, widely employed in other related works, to the less used Random Forest, trained using a low dimension representation of the input, the so called embeddings. We also propose a novel technique to combine different predictions, that exploit the uncertainty of the single model to assign it a proper weight during the aggregation phase. Although we weren’t able to surpass the performance of some related works, we obtained a mean AUROC of 0.902 over five different pathologies, showing that good results can be achieved also by models that are computationally more efficient and require much less time to be trained with respect to CNNs.
